\documentclass{chi2009}
\usepackage{times}
\usepackage{url}
\usepackage{graphics}
\usepackage{color}
\usepackage[pdftex]{hyperref}
\hypersetup{%
pdftitle={Your Title},
pdfauthor={Your Authors},
pdfkeywords={your keywords},
bookmarksnumbered,
pdfstartview={FitH},
colorlinks,
citecolor=black,
filecolor=black,
linkcolor=black,
urlcolor=black,
breaklinks=true,
}
\newcommand{\comment}[1]{}
\definecolor{Orange}{rgb}{1,0.5,0}
\newcommand{\todo}[1]{\textsf{\textbf{\textcolor{Orange}{[[#1]]}}}}

\pagenumbering{arabic}  % Arabic page numbers for submission.  Remove this line to eliminate page numbers for the camera ready copy

\begin{document}
% to make various LaTeX processors do the right thing with page size
\special{papersize=8.5in,11in}
\setlength{\paperheight}{11in}
\setlength{\paperwidth}{8.5in}
\setlength{\pdfpageheight}{\paperheight}
\setlength{\pdfpagewidth}{\paperwidth}

% use this command to override the default ACM copyright statement 
% (e.g. for preprints). Remove for camera ready copy.
\toappear{Submitted for review to CHI 2009.}

\title{Using AI and Social Media to analyse Depression}
\numberofauthors{2}
\author{
  \alignauthor Harun Gunasekaran\\
    \affaddr{University of Massachusetts}\\
    \affaddr{Lowell}\\
    \email{harun\_gunaseakran@student.uml.edu}
  \alignauthor Author 2\\
    \affaddr{Affiliation}\\
    \affaddr{Affiliation}\\
    \email{author2@b.com}
}

\maketitle

\begin{Abstract}
  In this project ,we implement three modules which will work with 
 each other inorder to analyse and state a person is 
 depressed or not. And the main primary objective of the project is to show
 the percentage to which the user is depressed, 
 thus the user can be directed to the available help services. By combining the results of all of three modules we can easily make decisions and thus by parsing the social media data into the NLP we can state whether a person is depressed or not.
\end{abstract}

\keywords{Natural Language Processing, Naive bayes Classifier, facepy} 



\section{Introduction}
  It is estimated that more than 50\% of teenager’s self harm and suicides were caused by depression[5] . It is also found that the number of human beings ending their life or struggling in their life because of stress and Depression will double in the next five years(‘4th edition of the Diagnostic and Statistical Manual of Mental Disorders (DSM-IV)’). Though it is widely argued what is the main causes of Depression, many live without even realising that they struggle from depression . This includes parents of adolescents who needs proper care and support.  And it is also found that more than 70\% of depressed prefer not to talk about their issues with others, since they are scared of being bullied and misjudged[4]. The recent research works have shown how social network and mental health correlate with each other , thus by analysing one’s social network preference can determine how mentally sound they are[3].
  And this is where we hope that using an Artificial Intelligence[1] as an intermediate to analyse the behaviour of the person can help and identify the depression level and thus can help them to find proper support. We intend to use the social network data from sites like facebook and also setting up questionnaires over a period of month to obtain the needed data . One of the main work done is similar field is by Microsoft research division[2] provided us with the needed base works, we found that the research work utilizes NLP to compare with the already defined dataset. And we intend to use similar approach but also use the Tags used by the user and associating them with the questionnaires.

\section{Project Description}

The project can be described interms of modules for easier understanding. The primary objective is to compare the modules and thus we can provide an effective result. The project uses the Natural language Toolkit along with SKLearn API to implement Naive Bayes Classifier. The NLTK provides with all the needed tools such as Tokenizer, tagger parts of speech tagger. and SKlearn API provides us with the needed classifier maodules. The effective but simple way for the project to succed we are gathering the user datas over a period of a month. Our objective was to analyses every weekend thus the major activities can be discovered. 


\section{Quetionare Module}
The first module of the project in which we ask user a set of five questions over a period of a month. This questions are defined by psycologiest to detect the depression . Inother words each question has its own weightage thus at the end of the five questions those who score more then 20 can be considered high risk and those who score between 15 to 20 as medium and less then 15 as low risk. This modules helps us to identify and erradicate the false positive while analysing the social media data. A typical set of quetionare will be as of : 
1. Have you slept properly this week?. 
2. How was your week interms of happiness?.
3. Have you ever felt sad the whole week?.
4. will you say your week was upto your satisfaction?

And the answers for this module will be of numerals 0 to 5 . Once the module executes for the first time , it will calculate the score as per the input of the user and categorise the user. If the user fall under High risk range his social media data will be analysed immediately . If he falls under medium category he will be made to take the test again after a week but still the social media data will be analysed for comparition.For this project we used a simple python based command prompt through which the user can interact with the module. 


\subsection{Social Media module}


Your paper's title, authors and affiliations should run across the
full width of the page in a single column 17.8 cm (7 in.) wide.  The
title should be in Helvetica 18-point bold; use Arial if Helvetica is
not available.  Authors' names should be in Times Roman 12-point bold,
and affiliations in Times Roman 12-point (note that Author and
Affiliation are defined Styles in this template file).

To position names and addresses, use a single-row table with invisible
borders, as in this document.  Alternatively, if only one address is
needed, use a centered tab stop to center all name and address text on
the page; for two addresses, use two centered tab stops, and so
on. For more than three authors, you may have to place some address
information in a footnote, or in a named section at the end of your
paper. Please use full international addresses and telephone dialing
prefixes.  Leave one 10-pt line of white space below the last line of
affiliations.

\subsection{Classifier module}

Every submission should begin with an abstract of about 150 words,
followed by a set of keywords. The abstract and keywords should be
placed in the left column of the first page under the left half of the
title. The abstract should be a concise statement of the problem,
approach and conclusions of the work described.  It should clearly
state the paper's contribution to the field of HCI.

The first set of keywords will be used to index the paper in the
proceedings. The second set are used to catalogue the paper in the ACM
Digital Library. The latter are entries from the ACM Classification
System~\cite{acm_categories}.  In general, it should only be necessary
to pick one or more of the H5 subcategories, see
http://www.acm.org/class/1998/H.5.html

\subsection{Integration of modules}

Please use a 10-point Times Roman font or, if this is unavailable,
another proportional font with serifs, as close as possible in
appearance to Times Roman 10-point. The Press 10-point font available
to users of Script is a good substitute for Times Roman. If Times
Roman is not available, try the font named Computer Modern Roman. On a
Macintosh, use the font named Times and not Times New Roman. Please
use sans-serif or non-proportional fonts only for special purposes,
such as headings or source code text.

\subsection{Analysis of Result}

Leave 3 cm (1.25 in.) of blank space for the copyright notice at the
bottom of the left column of the first page. In this template a
floating text box will automatically generate the required space.

\subsection{Discussion}

On pages beyond the first, start at the top of the page and continue
in double-column format.  The two columns on the last page should be
of equal length.

\subsection{Conclusion}


\subsection{References and Citations}

Use a numbered list of references at the end of the article, ordered
alphabetically by first author, and referenced by numbers in brackets
[2,4,5,7]. For papers from conference proceedings, include the title
of the paper and an abbreviated name of the conference (e.g., for
Interact 2003 proceedings, use Proc. Interact 2003). Do not include
the location of the conference or the exact date; do include the page
numbers if available. See the examples of citations at the end of this
document. Within this template file, use the References style for the
text of your citation.

Your references should be published materials accessible to the
public.  Internal technical reports may be cited only if they are
easily accessible (i.e., you provide the address for obtaining the
report within your citation) and may be obtained by any reader for a
nominal fee.  Proprietary information may not be cited. Private
communications should be acknowledged in the main text, not referenced
(e.g., ``[Robertson, personal communication]'').


\bibliographystyle{abbrv}
\bibliography{sample}

\end{document}
